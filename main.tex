%------------------------------------------------------------------------------
% Template file for the submission of papers to IUCr journals in LaTeX2e
% using the iucr document class
% Copyright 1999-2013 International Union of Crystallography
% Version 1.6 (28 March 2013)
%------------------------------------------------------------------------------

\documentclass[]{setup/iucr}              % DO NOT DELETE THIS LINE

     %-------------------------------------------------------------------------
     % Information about journal to which submitted
     %-------------------------------------------------------------------------
     \journalcode{A}              % Indicate the journal to which submitted
                                  %   A - Acta Crystallographica Section A
                                  %   B - Acta Crystallographica Section B
                                  %   C - Acta Crystallographica Section C
                                  %   D - Acta Crystallographica Section D
                                  %   E - Acta Crystallographica Section E
                                  %   F - Acta Crystallographica Section F
                                  %   J - Journal of Applied Crystallography
                                  %   M - IUCrJ
                                  %   S - Journal of Synchrotron Radiation

%%%%%%%%%%%%%%%%%%%%%%%%%%%%%%%%%%%%%%%%%%%%%%%%
% Encoding and fonts
%%%%%%%%%%%%%%%%%%%%%%%%%%%%%%%%%%%%%%%%%%%%%%%%
% Choose the output font encoding
\usepackage[T1]{fontenc}
% Choose the input encoding (how your source files are encoded)
\usepackage[utf8]{inputenc}

% Use the vector font Latin Modern which is going
% to be the default font in latex in the future.
% \usepackage{lmodern}
% \pdfmapfile{ua1r8r.map}

% Make latex understand and use the typographic
% rules of the language used in the document.
\usepackage[english, french]{babel}

%%%%%%%%%%%%%%%%%%%%%%%%%%%%%%%%%%%%%%%%%%%%%%%%
% Graphics (images, tables, ...)
%%%%%%%%%%%%%%%%%%%%%%%%%%%%%%%%%%%%%%%%%%%%%%%%

% Figures - TIKZ (Build figures inside LaTeX)
\usepackage{tikz}
\usetikzlibrary{arrows,shapes,positioning,shadows,trees}

%%%%%%%%%%%%%%%%%%%%%%%%%%%%%%%%%%%%%%%%%%%%%%%%
% Mathematics
%%%%%%%%%%%%%%%%%%%%%%%%%%%%%%%%%%%%%%%%%%%%%%%%
% Defines new environments such as equation,
% align and split 
% Provides additional features to amsmath
\usepackage[intlimits]{mathtools}
\usepackage{relsize}
% Adds new math symbols
\usepackage{amssymb}

% Use theorems in your document
% The ntheorem package is also used for the example environment
% When using thmmarks, amsmath must be an option as well. Otherwise \eqref doesn't work anymore.
\usepackage[framed,amsmath,thmmarks]{ntheorem}

%%%%%%%%%%%%%%%%%%%%%%%%%%%%%%%%%%%%%%%%%%%%%%%%
% Physics and chemistry
%%%%%%%%%%%%%%%%%%%%%%%%%%%%%%%%%%%%%%%%%%%%%%%%
% Provide SI units macros, to use proper formatting that respects SI standards
\usepackage{siunitx}
\sisetup{decimalsymbol=comma}
\sisetup{binary-units}
\sisetup{detect-weight}
\sisetup{inter-unit-product=\ensuremath{{} \cdot {}}}
\usepackage{cancel}

\usepackage[version=4]{mhchem}

% Macros are defined in setup/macros.tex

%%%%%%%%%%%%%%%%%%%%%%%%%%%%%%%%%%%%%%%%%%%%%%%%
% Misc
%%%%%%%%%%%%%%%%%%%%%%%%%%%%%%%%%%%%%%%%%%%%%%%%

% Enables the use FiXme refferences. Syntax: \fxnote{...} %%%
% When the document is in "final" version instead of "draft",
% an error will occur for every FiXme under compilation.
\usepackage[french, silent]{fixme}
\fxusetheme{color}
\fxsetup{layout=marginclue,footnote}
 % Packages inclusion
%%%%%%%%%%%%%%%%%%%%%%%%%%%%%%%%%%%%%%%%%%%%%%%%%%%%%
%             UNITS, EQUATIONS AND TEXT             %
%%%%%%%%%%%%%%%%%%%%%%%%%%%%%%%%%%%%%%%%%%%%%%%%%%%%%

\newcommand{\eq}[2]{\si{#1} &= \si{#2}}
\newcommand{\iaoi}{&& &\Updownarrow&&}  % 'If and only if' sign printed vertically between two equations
\newcommand{\eqOne}[2]{\si{#1} &= \si{#2} &\nonumber\\}
\newcommand{\eqTwo}[1]{&\ \ \ \ \si{#1}&}
% Text:
\newcommand{\tx}[1]{\text{#1}}
\newcommand{\deriv}{\mathrm{d}}

\newcommand{\noun}[1]{\textsc{#1}}
\newcommand*{\rom}[1]{\expandafter\@slowromancap\romannumeral #1@}

% Matrices, vectors & quaternions
%\renewcommand{\vec}[1]{\boldsymbol{\bf{#1}}}
\newcommand{\mat}[1]{#1}
\newcommand{\quat}[1]{\boldsymbol{\mathbf{#1}}}
\newcommand{\quatprod}{\otimes}

% Useful for declaration of math terms signification, right after an equation
\renewcommand*\descriptionlabel[1]{\hspace\leftmargin$#1$}

% Vertical line in equations ie. |_x=y (whereTwo stacks two equalities at the line)
\newcommand{\where}[1]{ \left.\rule{0cm}{.5cm}\right\vert\rule{0cm}{.4cm}_{\substack{\rule{0cm}{.15cm}\\ \si{#1} }} }
\newcommand{\whereTwo}[2]{ \left.\rule{0cm}{.67cm}\right\vert\rule{0cm}{.5cm}_{\substack{\si{#1} \rule{0cm}{.19cm}\\\vspace{-.1cm}\\ \si{#2}}} }

\newcommand{\sample}[1]{\ensuremath{#1}}

\makeatletter
  % Reaction equations environments
  \newcommand\reaction@[1]{\begin{equation}\ce{#1}\end{equation}}
  \newcommand\reaction@nonumber[1]{\begin{equation*}\ce{#1}\end{equation*}}
  \newcommand\reaction{\@ifstar{\reaction@nonumber}{\reaction@}}

  % Layer stacks and redox couples, useful in physics
  \newcommand*\stackslash{\text{/}\allowbreak}
  \newcommand*\stackhyphen{\text{-}\allowbreak}
  \newcommand*\stackpipe{\text{|}\allowbreak}
  \newcommand\stack[1]{%
    \cesplit{%
      {\/}{\c{stackslash}}%
      {-}{\c{stackhyphen}}%
      {\|}{\ \c{stackpipe}\ }%
    }{#1}%
  }
  \newcommand*\dopingcolumn{\text{:}\allowbreak}
  \newcommand\doped[1]{%
    \cesplit{%
      {:}{\c{dopingcolumn}}%
    }{#1}%
  }
\makeatother

% Hermann-Mauguin notation (for cristallographic symmetries)
\ExplSyntaxOn
\NewDocumentCommand{\hmn}{m}
 {
  \ensuremath
   {
    \hermannmauguin_group:n { #1 }
   }
 }

\tl_new:N \l_hermannmauguin_input_tl
\tl_new:N \l_hermannmauguin_output_tl

\cs_new_protected:Nn \hermannmauguin_group:n
 {
  \tl_set:Nn \l_hermannmauguin_input_tl { #1 }
  \tl_clear:N \l_hermannmauguin_output_tl
  \tl_map_inline:Nn \l_hermannmauguin_input_tl
   {
    \__hermannmauguin_item:n { ##1 }
   }
  \! % kill the first \,
  \tl_use:N \l_hermannmauguin_output_tl
 }

\cs_new_protected:Nn \__hermannmauguin_item:n
 {
  \str_case:nnF { #1 }
   {
    {*}{ \__hermannmauguin_put:n { \__hermannmauguin_inverse:Nn } }
    {-}{ \__hermannmauguin_put:n { \__hermannmauguin_overline:Nn } }
    {i}{ \__hermannmauguin_put:n { \,\infty } }
   }
   { \__hermannmauguin_put:n { \, {#1} } }
 }

\cs_new_protected:Nn \__hermannmauguin_put:n
 {
  \tl_put_right:Nn \l_hermannmauguin_output_tl { #1 }
 }

\cs_new_protected:Nn \__hermannmauguin_overline:Nn
 {% #1 should be \,; #2 is the number to operate on
  #1 \mkern1mu\overline{\mkern-1mu#2\mkern-1mu}\mkern1mu
 }
\cs_new_protected:Nn \__hermannmauguin_inverse:Nn
 {% #1 should be \,; #2 is the number to operate on
  #1 \frac{ #2 } { m }
 }
\ExplSyntaxOff

%%%%%%%%%%%%%%%%%%%%%%%%%%%%%%%%%%%%%%%%%%%%%%%%%%%%%
%                 TIKZ SETTINGS                     %
%%%%%%%%%%%%%%%%%%%%%%%%%%%%%%%%%%%%%%%%%%%%%%%%%%%%%
\usetikzlibrary{arrows.meta}
\tikzset{
  block/.style  = {draw, thick, rectangle,
                     minimum height = 2.1em,
                     minimum width = 1.7em},
  sum/.style    = {draw, circle, inner sep=1.5pt},
  basic/.style  = {draw, text width=190pt, minimum height=45pt,
                   font=\sffamily, inner sep=0pt},
}

% Node macro to build trees
\newcommand\mynode[5][]{
  \node[#1] (#2)
  {\parbox{20pt}{%
      \includegraphics[width=45pt]{#3}}%
    \parbox{25pt}{\mbox{}}%  
    \parbox{\dimexpr190pt-30pt\relax}{#4\\[.4ex]#5}%
  };
}

%%%%%%%%%%%%%%%%%%%%%%%%%%%%%%%%%%%%%%%%%%%%%%%%%%%%%
%                  REFERENCES                       %
%%%%%%%%%%%%%%%%%%%%%%%%%%%%%%%%%%%%%%%%%%%%%%%%%%%%%

%Chapter
\newcommand{\Chapref}[1]{\emph{Chapitre \ref{#1}}}
\newcommand{\chapref}[1]{\emph{chapitre \ref{#1}}}
%Section
\newcommand{\Secref}[1]{\emph{Section \ref{#1}}}
\newcommand{\secref}[1]{\emph{section \ref{#1}}}
%subSection
\newcommand{\Subsecref}[1]{\emph{Sous-section \ref{#1}}}
\newcommand{\subsecref}[1]{\emph{sous-section \ref{#1}}}
%Appendix
\newcommand{\Appref}[1]{\emph{Annexe \ref{#1}}}
\newcommand{\appref}[1]{\emph{annexe \ref{#1}}}
%Listings
\newcommand{\Coderef}[1]{\emph{Listing: \ref{#1}}}
\newcommand{\coderef}[1]{\emph{listing: \ref{#1}}}
%Figure:
\newcommand{\Figref}[1]{\emph{Figure \ref{#1}}}
\newcommand{\figref}[1]{\emph{figure \ref{#1}}}
\newcommand{\FigrefTwo}[2]{\emph{Figures \ref{#1}} et \emph{\ref{#2}}}
\newcommand{\figrefTwo}[2]{\emph{figures \ref{#1}} et \emph{\ref{#2}}}
%Table:
\newcommand{\Tableref}[1]{\emph{Table \ref{#1}}}
\newcommand{\tableref}[1]{\emph{table \ref{#1}}}

%Expressions:
\newcommand{\Expr}[1]{\emph{Expression (\ref{#1})}}
\newcommand{\expr}[1]{\emph{expression (\ref{#1})}}

%Equations:
%1 equation:
\newcommand{\Eqref}[1]{\emph{Équation (\ref{#1})}}
\renewcommand{\eqref}[1]{\emph{équation (\ref{#1})}}
%2 equations:
\newcommand{\EqrefTwo}[2]{\emph{Équations (\ref{#1})} et \emph{(\ref{#2})}}
\newcommand{\eqrefTwo}[2]{\emph{équations (\ref{#1})} et \emph{(\ref{#2})}}
%3 equations:
\newcommand{\EqrefThree}[3]{\emph{Equation (\ref{#1})}, \emph{(\ref{#2})} and \emph{(\ref{#3})}}
\newcommand{\eqrefThree}[3]{\emph{equation (\ref{#1})}, \emph{(\ref{#2})} and \emph{(\ref{#3})}}
%4 equations:
\newcommand{\EqrefFour}[4]{\emph{Equation (\ref{#1})}, \emph{(\ref{#2})}, \emph{(\ref{#3})} and \emph{(\ref{#4})}}
\newcommand{\eqrefFour}[4]{\emph{equation (\ref{#1})}, \emph{(\ref{#2})}, \emph{(\ref{#3})} and \emph{(\ref{#4})}}
%5 equations:
\newcommand{\EqrefFive}[5]{\emph{Equation (\ref{#1})}, \emph{(\ref{#2})}, \emph{(\ref{#3})}, \emph{(\ref{#4})} and \emph{(\ref{#5})}}
\newcommand{\eqrefFive}[5]{\emph{equation (\ref{#1})}, \emph{(\ref{#2})}, \emph{(\ref{#3})}, \emph{(\ref{#4})} and \emph{(\ref{#5})}}
%6 equations:
\newcommand{\EqrefSix}[6]{\emph{Equation (\ref{#1})}, \emph{(\ref{#2})}, \emph{(\ref{#3})}, \emph{(\ref{#4})}, \emph{(\ref{#5})} and \emph{(\ref{#6})}}
\newcommand{\eqrefSix}[6]{\emph{equation (\ref{#1})}, \emph{(\ref{#2})}, \emph{(\ref{#3})}, \emph{(\ref{#4})}, \emph{(\ref{#5})} and \emph{(\ref{#6})}}
%7 equations:
\newcommand{\EqrefSeven}[7]{\emph{Equation (\ref{#1})}, \emph{(\ref{#2})}, \emph{(\ref{#3})}, \emph{(\ref{#4})}, \emph{(\ref{#5})}, \emph{(\ref{#6})} and \emph{(\ref{#7})}}
\newcommand{\eqrefSeven}[7]{\emph{equation (\ref{#1})}, \emph{(\ref{#2})}, \emph{(\ref{#3})}, \emph{(\ref{#4})}, \emph{(\ref{#5})}, \emph{(\ref{#6})} and \emph{(\ref{#7})}} % My new macros

\begin{document}                  % DO NOT DELETE THIS LINE

     %-------------------------------------------------------------------------
     % The introductory (header) part of the paper
     %-------------------------------------------------------------------------

     % The title of the paper. Use \shorttitle to indicate an abbreviated title
     % for use in running heads (you will need to uncomment it).

\title{Détermination de la structure cristallographique de la pyrite}
%\shorttitle{Short Title}

     % Authors' names and addresses. Use \cauthor for the main (contact) author.
     % Use \author for all other authors. Use \aff for authors' affiliations.
     % Use lower-case letters in square brackets to link authors to their
     % affiliations; if there is only one affiliation address, remove the [a].

\cauthor{Julien}{Bréhin}{julien.brehin@etu.upmc.fr}{}
\author{Mourad}{Mezaguer}

\aff{M2 Sciences des Matériaux et Nano-Objets, Sorbonne Université, Paris, FRANCE}

     % Use \shortauthor to indicate an abbreviated author list for use in
     % running heads (you will need to uncomment it).

%\shortauthor{Soape, Author and Doe}

     % Use \vita if required to give biographical details (for authors of
     % invited review papers only). Uncomment it.

%\vita{Author's biography}

     % Keywords (required for Journal of Synchrotron Radiation only)
     % Use the \keyword macro for each word or phrase, e.g. 
     % \keyword{X-ray diffraction}\keyword{muscle}

%\keyword{keyword}

     % PDB and NDB reference codes for structures referenced in the article and
     % deposited with the Protein Data Bank and Nucleic Acids Database (Acta
     % Crystallographica Section D). Repeat for each separate structure e.g
     % \PDBref[dethiobiotin synthetase]{1byi} \NDBref[d(G$_4$CGC$_4$)]{ad0002}

%\PDBref[optional name]{refcode}
%\NDBref[optional name]{refcode}

\maketitle                        % DO NOT DELETE THIS LINE

\begin{synopsis}
Supply a synopsis of the paper for inclusion in the Table of Contents.
\end{synopsis}

\begin{abstract}
Nous allons, dans ce TP, étudier un minéral ayant une notoriété historique,
à savoir la Pyrite \ce{FeS2} (nommé aussi l'or des fous),
par diffraction des rayons X. Le matériau se cristallise dans le système
cubique, nous allons chercher d'abord à analyser le diagramme de diffraction
par la méthode de \noun{Debye-Scherrer} afin d'extraire
le paramètre de maille et les différentes règles d'extinctions ; à
partir desquelles nous pouvons cerner le groupe ponctuel à lequel
apparient notre minéral, puis par la méthode de \noun{Laue}. 
\end{abstract}


     %-------------------------------------------------------------------------
     % The main body of the paper
     %-------------------------------------------------------------------------
     % Now enter the text of the document in multiple \section's, \subsection's
     % and \subsubsection's as required.

\section{Introduction}

La pyrite, de formule chimique \ce{FeS2}, est plus connue sous le nom de «~pierre à feu~». Elle doit d'ailleurs son nom à sa capacité à produire des étincelles lorsqu'on la frotte ou qu'on la soumet à des chocs avec certains autres matériaux. Longtemps surnommée «~l'or des fous~» de par sa teinte et son aspect qui ont sûrement trompé plus d'un chercheur d'or, la pyrite est aujourd'hui particulièrement utilisée en joaillerie ou dans l'industrie chimique, pour la fabrication d'acide sulfurique. \fxnote{Citation needed}\\
On la trouve naturellement dans des gisements à des endroits très divers, en Europe comme en Amérique (lieu d'une «~ruée vers l'or~» au XIXe siècle) souvent mélangée à d'autres composés proches où le fer est substitué par d'autres métaux.\\
Les monocristaux de \ce{FeS2} se présentent sous deux apparences différentes~: cubique ou pentagono-dodécaédrique.

% Forme des monocristaux
% 

\section{Section title}

Text text text text text text text text text text text text text text
text text text text text text text.

\subsection{Title}

Text text text text text text text text text text text text text text
text text text text text text text.

\subsubsection{Title}

Text text text text text text text text text text text text text text
text text text text text text text.


     % Appendices appear after the main body of the text. They are prefixed by
     % a single \appendix declaration, and are then structured just like the
     % body text.

\appendix
\section{Appendix title}

Text text text text text text text text text text text text text text
text text text text text text text.

\subsection{Title}

Text text text text text text text text text text text text text text
text text text text text text text.

\subsubsection{Title}

Text text text text text text text text text text text text text text
text text text text text text text.


     %-------------------------------------------------------------------------
     % The back matter of the paper - acknowledgements and references
     %-------------------------------------------------------------------------

     % Acknowledgements come after the appendices

\ack{Acknowledgements}

     % References are at the end of the document, between \begin{references}
     % and \end{references} tags. Each reference is in a \reference entry.

% \begin{references}
% \reference{Author, A. \& Author, B. (1984). \emph{Journal} \textbf{Vol}, 
% first page--last page.}
% \end{references}
% \cite{knuth84}

%% Note added by Overleaf: If using bibtex, remove the "references" environment above, and uncomment the following lines.
\bibliographystyle{setup/iucr}
\referencelist{setup/sources}

     %-------------------------------------------------------------------------
     % TABLES AND FIGURES SHOULD BE INSERTED AFTER THE MAIN BODY OF THE TEXT
     %-------------------------------------------------------------------------

     % Simple tables should use the tabular environment according to this
     % model

\begin{table}
\caption{Caption to table}
\begin{tabular}{llcr}      % Alignment for each cell: l=left, c=center, r=right
 HEADING    & FOR        & EACH       & COLUMN     \\
\hline
 entry      & entry      & entry      & entry      \\
 entry      & entry      & entry      & entry      \\
 entry      & entry      & entry      & entry      \\
\end{tabular}
\end{table}

     % Postscript figures can be included with multiple figure blocks

\begin{figure}
\caption{Caption describing figure.}
\includegraphics{figures/fig1}
\end{figure}


\end{document}                    % DO NOT DELETE THIS LINE
%%%%%%%%%%%%%%%%%%%%%%%%%%%%%%%%%%%%%%%%%%%%%%%%%%%%%%%%%%%%%%%%%%%%%%%%%%%%%%
