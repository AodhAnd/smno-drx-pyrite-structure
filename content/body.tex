\section{Introduction}

La pyrite, de formule chimique \ce{FeS2}, est plus connue sous le nom de «~pierre à feu~». Elle doit d'ailleurs son nom à sa capacité à produire des étincelles lorsqu'on la frotte ou qu'on la soumet à des chocs avec certains autres matériaux. Longtemps surnommée «~l'or des fous~» de par sa teinte et son aspect qui ont sûrement trompé plus d'un chercheur d'or, la pyrite est aujourd'hui particulièrement utilisée en joaillerie ou dans l'industrie chimique, pour la fabrication d'acide sulfurique. \fxnote{Citation needed}\\
On la trouve naturellement en quantités relativement abondantes dans des gisements à des endroits très divers, en Europe comme en Amérique (lieu d'une «~ruée vers l'or~» au XIXe siècle). Bien qu'elle soit souvent mélangée, par exemple à d'autres composés proches où le fer est substitué par d'autres métaux, il est également possible d'extraire des monocristaux de \ce{FeS2} de taille assez importante sous forme de cubes surprenamment très lisses ou de sous forme de pentagono-dodécaèdres.

La pyrite fait donc partie des minéraux que l'on connaît et que l'on a utilisé depuis assez longtemps pour ses propriétés macroscopiques, avant-même de comprendre ses propriétés mésoscopiques et \textit{a fortiori} nanoscopiques.\\
Dans cet article, on s'attachera à déterminer la structure cristallographique complète de la pyrite, en partant d'une étude macroscopique et jusqu'à retrouver son groupe d'espace et ses paramètres de maille.
% But de l'article : déterminer la structure cristallographique complète de la pyrite
% Méthodes expérimentales DRX (Laue + SOLEIL/CRISTAL)
% Étude préliminaire : Forme des monocristaux
% Éléments de symétrie et groupe ponctuel du cube
% Éléments de symétrie et groupe ponctuel du pentagono-dodécaèdre
% Observation des éléments de symétrie du cliché de Laue
% Détermination du groupe de Laue -> comparaison avec le groupe ponctuel supposé et limites des clichés de Laue
% => Diffractogramme sur synchrotron (justifications)
% Tableau des 10 premières raies : numéro des raies (h²+k²+l²), indices (hkl), angles 2*theta, d_hkl, d_i+1/d_i, paramètre de maille
% Lecture des positions angulaires (2*theta) des 10 premières raies
% Calcul des distances inter-réticulaires (Loi de Bragg)
% Calcul des valeurs de rapport entre distances inter-réticulaires (di+1/di)
% Intérêt de ces rapports et comparaison des valeurs expérimentales avec des rapports calculés
% Obtention des indices hkl pour chacune des raies et détermination du mode de réseau (primitif, sinon il y aurait plus d'extinctions -> facteur de structure décomposé en mode+motif)
% Détermination du paramètre de maille cubique à partir des d_hkl (formule)
% Moyenne et écart-type de a
% Remarque sur le rapport à valeur étrange et explication sur la raie manquante
% Autres directions manquantes sur le diffractogramme
% Parité des directions manquantes et comparaison avec les IToC (annexe 3)
% Déduction du groupe d'espace
% Explication extinctions élément de symétrie (élément translatoire => facteur de structure)
% Positions atomiques (motif), sites de Wyckoff, ...
% Modèle VESTA
% Comparaison diagramme poudres VESTA vs expérimental SOLEIL
% Conclusion et perspectives

\section{Section title}

Text text text text text text text text text text text text text text
text text text text text text text.

\subsection{Title}

Text text text text text text text text text text text text text text
text text text text text text text.

\subsubsection{Title}

Text text text text text text text text text text text text text text
text text text text text text text.
