%%%%%%%%%%%%%%%%%%%%%%%%%%%%%%%%%%%%%%%%%%%%%%%%%%%%%
%             UNITS, EQUATIONS AND TEXT             %
%%%%%%%%%%%%%%%%%%%%%%%%%%%%%%%%%%%%%%%%%%%%%%%%%%%%%

\newcommand{\eq}[2]{\si{#1} &= \si{#2}}
\newcommand{\iaoi}{&& &\Updownarrow&&}  % 'If and only if' sign printed vertically between two equations
\newcommand{\eqOne}[2]{\si{#1} &= \si{#2} &\nonumber\\}
\newcommand{\eqTwo}[1]{&\ \ \ \ \si{#1}&}
% Text:
\newcommand{\tx}[1]{\text{#1}}
\newcommand{\deriv}{\mathrm{d}}

\newcommand{\noun}[1]{\textsc{#1}}
\newcommand*{\rom}[1]{\expandafter\@slowromancap\romannumeral #1@}

% Matrices, vectors & quaternions
%\renewcommand{\vec}[1]{\boldsymbol{\bf{#1}}}
\newcommand{\mat}[1]{#1}
\newcommand{\quat}[1]{\boldsymbol{\mathbf{#1}}}
\newcommand{\quatprod}{\otimes}

% Useful for declaration of math terms signification, right after an equation
\renewcommand*\descriptionlabel[1]{\hspace\leftmargin$#1$}

% Vertical line in equations ie. |_x=y (whereTwo stacks two equalities at the line)
\newcommand{\where}[1]{ \left.\rule{0cm}{.5cm}\right\vert\rule{0cm}{.4cm}_{\substack{\rule{0cm}{.15cm}\\ \si{#1} }} }
\newcommand{\whereTwo}[2]{ \left.\rule{0cm}{.67cm}\right\vert\rule{0cm}{.5cm}_{\substack{\si{#1} \rule{0cm}{.19cm}\\\vspace{-.1cm}\\ \si{#2}}} }

\newcommand{\sample}[1]{\ensuremath{#1}}

\makeatletter
  % Reaction equations environments
  \newcommand\reaction@[1]{\begin{equation}\ce{#1}\end{equation}}
  \newcommand\reaction@nonumber[1]{\begin{equation*}\ce{#1}\end{equation*}}
  \newcommand\reaction{\@ifstar{\reaction@nonumber}{\reaction@}}

  % Layer stacks and redox couples, useful in physics
  \newcommand*\stackslash{\text{/}\allowbreak}
  \newcommand*\stackhyphen{\text{-}\allowbreak}
  \newcommand*\stackpipe{\text{|}\allowbreak}
  \newcommand\stack[1]{%
    \cesplit{%
      {\/}{\c{stackslash}}%
      {-}{\c{stackhyphen}}%
      {\|}{\ \c{stackpipe}\ }%
    }{#1}%
  }
  \newcommand*\dopingcolumn{\text{:}\allowbreak}
  \newcommand\doped[1]{%
    \cesplit{%
      {:}{\c{dopingcolumn}}%
    }{#1}%
  }

\makeatother


%%%%%%%%%%%%%%%%%%%%%%%%%%%%%%%%%%%%%%%%%%%%%%%%%%%%%
%                 TIKZ SETTINGS                     %
%%%%%%%%%%%%%%%%%%%%%%%%%%%%%%%%%%%%%%%%%%%%%%%%%%%%%
\usetikzlibrary{arrows.meta}
\tikzset{
  block/.style  = {draw, thick, rectangle,
                     minimum height = 2.1em,
                     minimum width = 1.7em},
  sum/.style    = {draw, circle, inner sep=1.5pt},
  basic/.style  = {draw, text width=190pt, minimum height=45pt,
                   font=\sffamily, inner sep=0pt},
}

% Node macro to build trees
\newcommand\mynode[5][]{
  \node[#1] (#2)
  {\parbox{20pt}{%
      \includegraphics[width=45pt]{#3}}%
    \parbox{25pt}{\mbox{}}%  
    \parbox{\dimexpr190pt-30pt\relax}{#4\\[.4ex]#5}%
  };
}

%%%%%%%%%%%%%%%%%%%%%%%%%%%%%%%%%%%%%%%%%%%%%%%%%%%%%
%                  REFERENCES                       %
%%%%%%%%%%%%%%%%%%%%%%%%%%%%%%%%%%%%%%%%%%%%%%%%%%%%%

%Chapter
\newcommand{\Chapref}[1]{\emph{Chapitre \ref{#1}}}
\newcommand{\chapref}[1]{\emph{chapitre \ref{#1}}}
%Section
\newcommand{\Secref}[1]{\emph{Section \ref{#1}}}
\newcommand{\secref}[1]{\emph{section \ref{#1}}}
%subSection
\newcommand{\Subsecref}[1]{\emph{Sous-section \ref{#1}}}
\newcommand{\subsecref}[1]{\emph{sous-section \ref{#1}}}
%Appendix
\newcommand{\Appref}[1]{\emph{Annexe \ref{#1}}}
\newcommand{\appref}[1]{\emph{annexe \ref{#1}}}
%Listings
\newcommand{\Coderef}[1]{\emph{Listing: \ref{#1}}}
\newcommand{\coderef}[1]{\emph{listing: \ref{#1}}}
%Figure:
\newcommand{\Figref}[1]{\emph{Figure \ref{#1}}}
\newcommand{\figref}[1]{\emph{figure \ref{#1}}}
\newcommand{\FigrefTwo}[2]{\emph{Figures \ref{#1}} et \emph{\ref{#2}}}
\newcommand{\figrefTwo}[2]{\emph{figures \ref{#1}} et \emph{\ref{#2}}}
%Table:
\newcommand{\Tableref}[1]{\emph{Tableau \ref{#1}}}
\newcommand{\tableref}[1]{\emph{tableau \ref{#1}}}

%Expressions:
\newcommand{\Expr}[1]{\emph{Expression (\ref{#1})}}
\newcommand{\expr}[1]{\emph{expression (\ref{#1})}}

%Equations:
%1 equation:
\newcommand{\Eqref}[1]{\emph{Équation (\ref{#1})}}
\renewcommand{\eqref}[1]{\emph{équation (\ref{#1})}}
%2 equations:
\newcommand{\EqrefTwo}[2]{\emph{Équations (\ref{#1})} et \emph{(\ref{#2})}}
\newcommand{\eqrefTwo}[2]{\emph{équations (\ref{#1})} et \emph{(\ref{#2})}}
%3 equations:
\newcommand{\EqrefThree}[3]{\emph{Equation (\ref{#1})}, \emph{(\ref{#2})} and \emph{(\ref{#3})}}
\newcommand{\eqrefThree}[3]{\emph{equation (\ref{#1})}, \emph{(\ref{#2})} and \emph{(\ref{#3})}}
%4 equations:
\newcommand{\EqrefFour}[4]{\emph{Equation (\ref{#1})}, \emph{(\ref{#2})}, \emph{(\ref{#3})} and \emph{(\ref{#4})}}
\newcommand{\eqrefFour}[4]{\emph{equation (\ref{#1})}, \emph{(\ref{#2})}, \emph{(\ref{#3})} and \emph{(\ref{#4})}}
%5 equations:
\newcommand{\EqrefFive}[5]{\emph{Equation (\ref{#1})}, \emph{(\ref{#2})}, \emph{(\ref{#3})}, \emph{(\ref{#4})} and \emph{(\ref{#5})}}
\newcommand{\eqrefFive}[5]{\emph{equation (\ref{#1})}, \emph{(\ref{#2})}, \emph{(\ref{#3})}, \emph{(\ref{#4})} and \emph{(\ref{#5})}}
%6 equations:
\newcommand{\EqrefSix}[6]{\emph{Equation (\ref{#1})}, \emph{(\ref{#2})}, \emph{(\ref{#3})}, \emph{(\ref{#4})}, \emph{(\ref{#5})} and \emph{(\ref{#6})}}
\newcommand{\eqrefSix}[6]{\emph{equation (\ref{#1})}, \emph{(\ref{#2})}, \emph{(\ref{#3})}, \emph{(\ref{#4})}, \emph{(\ref{#5})} and \emph{(\ref{#6})}}
%7 equations:
\newcommand{\EqrefSeven}[7]{\emph{Equation (\ref{#1})}, \emph{(\ref{#2})}, \emph{(\ref{#3})}, \emph{(\ref{#4})}, \emph{(\ref{#5})}, \emph{(\ref{#6})} and \emph{(\ref{#7})}}
\newcommand{\eqrefSeven}[7]{\emph{equation (\ref{#1})}, \emph{(\ref{#2})}, \emph{(\ref{#3})}, \emph{(\ref{#4})}, \emph{(\ref{#5})}, \emph{(\ref{#6})} and \emph{(\ref{#7})}}